% Options for packages loaded elsewhere
\PassOptionsToPackage{unicode}{hyperref}
\PassOptionsToPackage{hyphens}{url}
\PassOptionsToPackage{dvipsnames,svgnames,x11names}{xcolor}
%
\documentclass[
]{article}
\usepackage{amsmath,amssymb}
\usepackage{iftex}
\ifPDFTeX
  \usepackage[T1]{fontenc}
  \usepackage[utf8]{inputenc}
  \usepackage{textcomp} % provide euro and other symbols
\else % if luatex or xetex
  \usepackage{unicode-math} % this also loads fontspec
  \defaultfontfeatures{Scale=MatchLowercase}
  \defaultfontfeatures[\rmfamily]{Ligatures=TeX,Scale=1}
\fi
\usepackage{lmodern}
\ifPDFTeX\else
  % xetex/luatex font selection
\fi
% Use upquote if available, for straight quotes in verbatim environments
\IfFileExists{upquote.sty}{\usepackage{upquote}}{}
\IfFileExists{microtype.sty}{% use microtype if available
  \usepackage[]{microtype}
  \UseMicrotypeSet[protrusion]{basicmath} % disable protrusion for tt fonts
}{}
\makeatletter
\@ifundefined{KOMAClassName}{% if non-KOMA class
  \IfFileExists{parskip.sty}{%
    \usepackage{parskip}
  }{% else
    \setlength{\parindent}{0pt}
    \setlength{\parskip}{6pt plus 2pt minus 1pt}}
}{% if KOMA class
  \KOMAoptions{parskip=half}}
\makeatother
\usepackage{xcolor}
\usepackage[margin=1in]{geometry}
\usepackage{graphicx}
\makeatletter
\def\maxwidth{\ifdim\Gin@nat@width>\linewidth\linewidth\else\Gin@nat@width\fi}
\def\maxheight{\ifdim\Gin@nat@height>\textheight\textheight\else\Gin@nat@height\fi}
\makeatother
% Scale images if necessary, so that they will not overflow the page
% margins by default, and it is still possible to overwrite the defaults
% using explicit options in \includegraphics[width, height, ...]{}
\setkeys{Gin}{width=\maxwidth,height=\maxheight,keepaspectratio}
% Set default figure placement to htbp
\makeatletter
\def\fps@figure{htbp}
\makeatother
\setlength{\emergencystretch}{3em} % prevent overfull lines
\providecommand{\tightlist}{%
  \setlength{\itemsep}{0pt}\setlength{\parskip}{0pt}}
\setcounter{secnumdepth}{-\maxdimen} % remove section numbering
\ifLuaTeX
  \usepackage{selnolig}  % disable illegal ligatures
\fi
\usepackage{bookmark}
\IfFileExists{xurl.sty}{\usepackage{xurl}}{} % add URL line breaks if available
\urlstyle{same}
\hypersetup{
  pdftitle={Tutoriel de sauvegarde de ses données de recherche},
  pdfauthor={Louis Manière - Université de Tours},
  colorlinks=true,
  linkcolor={Maroon},
  filecolor={Maroon},
  citecolor={Blue},
  urlcolor={blue},
  pdfcreator={LaTeX via pandoc}}

\title{Tutoriel de sauvegarde de ses données de recherche}
\author{Louis Manière - Université de Tours}
\date{}

\begin{document}
\maketitle

\section{Sauvegarder ses données --
Tutoriel}\label{sauvegarder-ses-donnuxe9es-tutoriel}

\subsection{Les disques réseaux de l'université
💾}\label{les-disques-ruxe9seaux-de-luniversituxe9}

Ces serveurs sont basés au Plat d'Etain avec une copie sur le campus de
Grandmont. Deux espaces sont disponibles, l'espaces personnel (U: pour
les utilisateurs de Windows) et l'espace laboratoire (T:). Il y a du
versionnement sur ce stockage, on peut actuellement remonter jusqu'à une
semaine en arrière. Pour l'instant il n'y a pas de limite d'espace
disque mais ce n'est évidemment pas infinie. Ainsi, comme tout stockage
de données, la directive est à la retenue (sélectionner ses dossiers à
sauvegarder, éviter les doublons, éviter les nombreuses versions d'un
même fichiers etc.).

Pour les doctorants, sur l'espace du laboratoire (T:) vous trouverez un
dossier \emph{Geosciences/Laboratoire GeHCO/Thèses} sur lequel vous
pouvez créer un dossier à votre nom et y placer vos travaux finalisés
(ou presque), soit vos données finalisées avec leur documentation, votre
manuscrit de thèse, vos articles, protocoles, plans etc. Une bonne
occasion de faire le ménage dans ses dossiers en fin de thèse\ldots{} et
pourquoi pas en profiter pour valoriser ses données.

On peut accéder à ces disques soit en étant branché en filaire à
l'université ou bien en VPN.

\begin{quote}
\textbf{La minute VPN}

\begin{itemize}
\item
  Pour les utilisateur de Windows : vous avez normalement déjà
  l'application \textbf{OpenVPN} d'installée. Pour le démarrer, ça se
  fait en 3 clicks, il y a un
  \href{https://utmedia.univ-tours.fr/permalink/v12664e6412c0z39x92t/iframe/}{tutoriel
  en ligne}.
\item
  Pour les utilisateurs MAC il faut utiliser l'application
  \textbf{Tunnelblick}, demandez à la DSI pour le mettre en route.
\end{itemize}
\end{quote}

\subsubsection{Utilisateurs de Windows
🪟}\label{utilisateurs-de-windows}

Vous trouvez ces disques dans l'explorateurs de fichiers classiques.

\includegraphics{img/disque_reseaux_win.png}

Le disque personnel (U:), si vous ne l'utilisez pas encore devrait être
vide. Vous pouvez y mettre vos données de recherche, cours etc. comme
vous un disque dur externe avec un simple copier/coller ou bien avec une
méthode automatisée (vous trouverez la procédure ci-dessous 👇).

\subsubsection{Utilisateurs MAC 🍎}\label{utilisateurs-mac}

Il vous faut aller dans \emph{Aller/Se connecter au serveur} pour créer
une nouvelle connexion. On va certainement vous demander votre mot de
passe de l'ordinateur.

\includegraphics{img/disque_reseaux_mac.png}

\begin{itemize}
\tightlist
\item
  Pour le disque personnel il faut mettre l'adresse avec son nom
  utilisateur :
  \href{smb://filerper.univ-tours.local/home/personnels/}{\emph{smb://filerper.univ-tours.local/home/personnels/}}\textbf{\emph{mon\_nom\_utilisateur}}
  Une fois rentré, le serveur va certainement vous demander votre
  identifiant et mot de passe de l'ENT cette fois.
\item
  Pour le disque laboratoire, il faut créer une nouvelle connexion avec
  cette adresse :
  \href{smb://filersciences.univ-tours.local/sciences$/Partage/Geosciences}{\emph{smb://filersciences.univ-tours.local/sciences\$/Partage/Geosciences}}
\end{itemize}

Vous devriez avoir accès à ces espaces directement avec la même démarche
\emph{Aller/Se connecter au serveur}.

\subsection{Automatiser ses sauvegardes
🤖}\label{automatiser-ses-sauvegardes}

Une solution simple pour mettre à jour sa sauvegarde quand on est à
l'université ou en VPN sans avoir à y penser est d'installer un petit
logiciel de sauvegarde comme
\href{https://www.2brightsparks.com/download-syncbackfree.html}{\textbf{SyncBackFree}}
pour Windows ou \href{https://freefilesync.org/}{\textbf{FreeFileSync}}
pour Windows/MAC/Linux.

La procédure est la même pour les deux logiciels. Vous désignez un
dossier \emph{source}, votre dossier à sauvegarder, et un dossier de
\emph{destination}, le dossier de sauvegarde sur votre disque réseau
personnel (U:) par exemple. Puis il s'agit de créer un \emph{profil} et
de paramétrer un mode de sauvegarde :

\begin{itemize}
\tightlist
\item
  \textbf{Sauvegarder} : Un backup ``simple'', le logiciel scan votre
  dossier \emph{source}, regarde les fichier modifiés ou manquants puis
  met à jour ces fichiers dans la \emph{destination}. Cela équivaut à
  prendre tout son dossier et faire un copier/coller en remplaçant les
  fichiers existants. Il n'y a pas de suppression dans la
  \emph{destination}, l'avantage est de conserver ses vieux fichiers
  supprimés de son ordinateur, l'inconvénient est que l'on conserve ses
  vieux fichiers supprimés de son ordinateurs\ldots{} 🤔.
\item
  \textbf{Synchroniser} : C'est le modèle des dossiers partagés et
  modifiables dans UTbox. Si on modifie un fichier synchronisé sur son
  poste, on modifie sa copie sur le disque réseau lors de la prochaine
  synchronisation. Si on modifie le fichier sur le disque réseau, sa
  version sur notre ordinateur sera modifié lors de la prochaine
  synchronisation. L'objectif est de toujours avoir la version la plus à
  jour sur son poste ou sur le serveur. C'est plus un outil de travail
  partagé qu'un outil de backup.
\item
  \textbf{Miroir} : Un backup qui s'occupe de toujours avoir la copie
  exacte entre la \emph{source} et la \emph{destination}. Le programme
  met donc à jour les fichiers ajoutés et modifiés de la
  \emph{destination} depuis la source mais supprime également ceux qui
  ont été mis à la corbeille. Cela présente l'avantage de nettoyer la
  sauvegarde en même temps que l'on nettoie son ordinateur des vieux
  fichiers.
\end{itemize}

Il est tout à fait possible de faire plusieurs profils pour différents
dossier avec des modes de sauvegarde différents.

\subsubsection{Un exemple sur
SyncBackFree}\label{un-exemple-sur-syncbackfree}

Je crée un profil avec un nom

\includegraphics{img/syncbackfree_profil.png}

Je choisis le mode de sauvegarde (ici un miroir)

\includegraphics{img/syncbackfree_miroir.png}

Je désigne un le type de dossier à synchroniser entre la \emph{source}
et la \emph{destination}, globalement vous pouvez laisser \emph{Unité
interne/externe}. Si votre dossier de sauvegarde est juste là pour
assurer la sécurité de vos données en cas de perte ou vol de votre
ordinateur et que vous allez donc rarement chercher des fichiers dans
votre sauvegarde, vous pouvez alors compresser la destination. Les
informaticiens de la DSI et la planète vous remercierons 🤗.

\includegraphics{img/syncbackfree_source_destination.png}

Et maintenant on finalise la paramétrisation :

\begin{itemize}
\tightlist
\item
  \emph{Source} : le chemin vers dossier à sauvegarder sur votre poste,
  ici \emph{C:/Users/maniere/Documents/dossier\_important}
\item
  \emph{Destination} : le chemin vers dossier que vous avez créer sur
  votre espace de sauvegarde qui va servir de copie, ici
  \emph{U:/dossier\_important\_mirror.zip} vers mon disque réseau
  personnel.
\end{itemize}

\begin{quote}
\textbf{Info compression !}

Si vous avez choisi de compresser votre dossier de \emph{destination} il
faut bien penser un dossier compressé plutôt qu'un dossier normal. Pour
cela, \emph{clic droit/nouveau/dossier compressé}.
\end{quote}

Vous pouvez mettre des filtres, des choix de sous-dossier etc., ici on
reste simple et on valide avec \emph{OK}.

Le programme vous diras certainement que ce n'est pas recommandé de
mettre des chemins de dossier avec des lettres (comme U:) mais il faut
utiliser le vrai chemin qui commence par
\emph{\textbackslash fileper\ldots{}}, vous êtes tout à fait d'accord
avec lui et répondez \emph{oui}!

Si vous créé un miroir vous aurez un avertissement, les fichiers de
\emph{destination} seront supprimés s'il ne sont pas sur \emph{source},
vous répondez \emph{OK}, c'est même l'intéret de faire un miroir.

\includegraphics{img/syncbackfree_chemin.png}

Vous pouvez enfin lancer votre première simulation.

\includegraphics{img/syncbackfree_first_scan.png}

En vert on trouve mes deux fichiers à créer. Tout est normal, on peut
\emph{Continuer la synchronisation}. On voit maintenant notre profil qui
montre que notre simulation est correct, on sélectionne notre profil et
on peut maintenant faire une \emph{Execution} pour effectuer les
modifications.

De cette façon on peut manuellement faire des mise à jour de notre
sauvegarde avec \emph{Exécution} puis \emph{Continuer l'éxécution}.

Mais on peut aussi :

\subsubsection{Planifier ses sauvegardes
📆}\label{planifier-ses-sauvegardes}

Pour cela sélectionner votre profil puis lancer \emph{Planification de
tâche}, vous aurez surement à entrer le mot de passe de votre ordinateur
la première fois.

Ici il s'agit de rentrer la fréquence de mise à jour de votre
sauvegarde. Ici je mets à jour ma sauvegarde chaque jour à 11h.

\includegraphics{img/syncbackfree_planificateur.png}

Il est bon de faire également quelques ajustements supplémentaires dans
l'onglet \emph{Paramètres} :

\begin{itemize}
\tightlist
\item
  J'éxecute uniquement la tâche si l'ordinateur est connecté.
\item
  Je choisis de ne pas réveiller l'ordinateur pour executer la backup,
  il faut que je soit présent sur l'ordinateur et non en veille pour que
  la sauvegarde s'effectue.
\item
  Je veux executer cette tâche dès que je peux si j'ai eu un échec, par
  exemple si je ne suis pas en VPN je veux que la sauvegarde se lance
  dès que je connecte au réseau de l'université ou au VPN.
\item
  J'arrête la sauvegarde si elle dure plus de 2h, cela doit se faire
  rapidement sauf peut-être la première sauvegarde où il y a beaucoup de
  chose à créer.
\end{itemize}

\includegraphics{img/syncbackfree_param_planification.png}

Une fois la planification paramétrée vous pouvez vérifier en ouvrant le
logiciel voir quand s'est effectuée la dernière sauvegarde et si cela
s'est bien déroulé.

\subsubsection{FreeFileSync pour les utilisateurs MAC et
Linux}\label{freefilesync-pour-les-utilisateurs-mac-et-linux}

Le principe reste le même que \textbf{SyncBackFree}, il y a juste
l'interface qui change et la planification qui est possible mais pas
aussi intégré au logiciel.

Les vidéos tutoriel sont disponible
\href{https://freefilesync.org/tutorials.php}{ici} pour faire un
mirroir, une sauvegarde simple ou de la planification. La planification
sur les différentes plateformes Windows, MAC ou Linux est également
détaillée dans la
\href{https://freefilesync.org/manual.php?topic=schedule-batch-jobs}{documentation}.

\subsection{UTbox}\label{utbox}

UTbox est un le cloud de l'université qui fonctionne en ligne via l'ENT
mais vos dossiers ainsi que ceux qui vous ont été partagé peuvent être
disponible sur votre ordinateur dans l'explorateur de fichier. Cela se
passe sur l'application \textbf{Nextcloud} déjà présent sur vos poste de
travail ou très facile à installer si ce n'est pas le cas.

\begin{quote}
\textbf{Le saviez-vous ?}

Un tutoriel de l'application Nextcloud et d'UTbox se trouve déjà (et
cela depuis le début!) sur votre espace UTbox. Rendez-vous sur
\textbf{UTBOX\_pas\_à\_pas.pdf} dans votre dossier UTbox \emph{UTbox -
Documentation utilisateur}
\end{quote}

Comme précisé dans le tutoriel de l'université, la synchronisation n'est
pas adaptée à une sauvegarde. UTbox est davantage un outil de travail
collaboratif que de sauvegarde des données.

\end{document}
